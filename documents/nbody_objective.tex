% This file is part of the Huey P Newton project.
% Copyright 2012 David W. Hogg (NYU).
% All rights reserved.

\documentclass[12pt,pdftex,preprint]{aastex}
\usepackage{amssymb,amsmath,mathrsfs}

\newcommand{\setof}[1]{\left\{{#1}\right\}}
\newcommand{\given}{\,|\,}
\newcommand{\bg}{\mathrm{bg}}
\newcommand{\dyn}{\mathrm{dyn}}
\newcommand{\ic}{\mathrm{I.C.}}
\newcommand{\normal}{\mathscr{N}}

\begin{document}

\title{A justified probabilistic objective \\ for comparing n-body theory to point data: \\ The Palomar 5 Stream}
\author{some combination of DWH, APW, AJ, AK, KVJ}

\begin{abstract}
In a wide range of astrophysical contexts, the dominant theoretical
methods are n-body simulations or methods that produce point sets.  In
an overlapping set of contexts, the dominant observational data sets
are also made up of---effectively---points in some high-dimensional
space: stars on the sky or in phase space, galaxies in redshift space,
and the like.  In principle, model testing ought to proceed by
comparing likelihoods or marginalized likelihoods or posterior
probabilities.  Here we make a very general proposal for constructing
a likelihood---a probability for the data given the output of the
n-body simulation.  It treats the data as being generated by a mixture
model built from the n-body points.  We employ this model---mixed with
a flexible background model---to fit a numerical model of a tidal
stream to observed stars in the vicinity of Palomar 5 Stream.  We find
that the Milky Way is XXX and the Stream is YYY.
\end{abstract}

\section{Generalities}

You have measured in a survey of some kind $N$ data points $Y_n$ in a
$d$-dimensional observable space $Y$.  The data points could be stars
and the space could be 6-dimensional phase space.  It could be
galaxies in 3-dimensional redshift space.  It could be quasars in just
2-dimensional celestial coordinates.  Or it could be stars in a much
higher dimensional space of phase space plus chemical indicators.

Along with each measurement comes a positive-definite $d\times d$
variance tensor $\sigma^2_n$, or a non-negative-definite inverse
variance tensor $\sigma^{-2}_n$.  This is a measure of the uncertainty
in the space, and also at the same time takes account of \emph{missing
  data}: Dimensions or directions which are unmeasured in the $Y$
space for point $Y_n$ are dimensions or directions with zero
eigenvalue in $\sigma^{-2}_n$.  That is, we can mix together stars
that have measured positions and velocities in with stars that just
have measured angular positions and with stars that have measured
proper motions; each of these different kinds of star has a different
rank to the $\sigma^{-2}_n$ tensor.

At the same time, you have a theory or model that can be computed with
a ``n-body'' simulation that generates $K$ points $y_k$ in that same
space.  The theory has a large vector or blob of parameters $\theta$.
If the theory is interesting and predictive, the positions $y_k$
depend non-trivially on the parameters $\theta$.

The fundamental idea is that if these model points are representative,
the data can be thought of as being \emph{generated by} the model
points.  Model point $y_k$ can generate a data point $Y_n$ with
probability $p(Y_n\given y_k,\theta,I)$; this would be the
single-data-point likelihood of the parameters and this one model
point $y_k$.  Because (usually) the identity $k$ of the model point
explaining data point $n$ is not of scientific interest, we want to
\emph{marginalize out} $k$ to get a probability $p(Y_n\given\theta,I)$
that is independent of $k$.

The math for all this is
\begin{eqnarray}
p(Y_n\given y_k,\theta,I) &=& \normal(Y_n\given y_k,\sigma^2_n+\Sigma^2)
\\
\theta &\equiv& \setof{\theta_\dyn, \theta_\ic, \Sigma^2}
\\
I &\equiv& \setof{K, \setof{\sigma^2_n}_{n=1}^N, \cdots}
\\
p(Y_n\given\theta,I) &=& \sum_{k=1}^K P_k\,p(Y_n\given y_k,\theta)
\\
1 &=& \sum_{k=1}^K P_k
\quad ,
\end{eqnarray}
where $I$ represents our inviolate prior information about the problem
(the number of points we are generating with the theory and the noise
model), $\normal(x\given m, V)$ is the $d$-dimensional Gaussian
distribution for $x$ given mean $m$ and variance tensor $V$,
$\Sigma^2$ is a smoothing variance in the $Y$-space to account for the
fact that the model points are infinitesimal, $\theta$ can be divided
into dynamical parameters (Milky Way potential parameters, for
example) and initial-condition parameters (the orbit and phase of the
body of Palomar 5, for example) and the smoothing, and the likelihood
can be marginalized over model points $k$ by summing weighted by the
prior probabilities $P_k$.  In the case that all the model points are
equally weighed or equally representative or predictive for the data,
\begin{eqnarray}
P_k &=& \frac{1}{K}
\quad .
\end{eqnarray}

HOGG... Now introduce a background model...
\begin{eqnarray}
P(\setof{Y_n}_{n=1}^N\given\theta,I) &=& \prod_{n=1}^N p(Y_n\given\theta,I)
\\
p(Y_n\given\theta,I) &=& P_\bg\,p(Y_n\given \bg,\theta) + \sum_{k=1}^K P_k\,p(Y_n\given y_k,\theta)
\\
\theta &\equiv& \setof{\theta_\dyn, \theta_\ic, \Sigma^2, \theta_\bg}
\\
1 &=& P_\bg + \sum_{k=1}^K P_k
\\
P_k &=& \frac{[1 - P_\bg]}{K}
\end{eqnarray}

\end{document}
