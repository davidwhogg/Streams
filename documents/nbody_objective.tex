% This file is part of the Huey P Newton project.
% Copyright 2012 David W. Hogg (NYU).
% All rights reserved.

\documentclass[12pt,pdftex,preprint]{aastex}

\newcommand{\setof}[1]{\left\{{#1}\right\}}
\newcommand{\given}{\,|\,}
\newcommand{\bg}{\mathrm{bg}}
\newcommand{\dyn}{\mathrm{dyn}}
\newcommand{\ic}{\mathrm{I.C.}}

\begin{document}

\title{A justified probabilistic objective \\ for comparing n-body theory to point data: \\ The Palomar 5 Stream}
\author{some combination of DWH, APW, AJ, AK, KVJ}

\begin{abstract}
In a wide range of astrophysical contexts, the dominant theoretical
methods are n-body simulations or methods that produce point sets.  In
an overlapping set of contexts, the dominant observational data sets
are also made up of---effectively---points in some high-dimensional
space: stars on the sky or in phase space, galaxies in redshift space,
and the like.  In principle, model testing ought to proceed by
comparing likelihoods or marginalized likelihoods or posterior
probabilities.  Here we make a very general proposal for constructing
a likelihood---a probability for the data given the output of the
n-body simulation.  It treats the data as being generated by a mixture
model built from the n-body points.  We employ this model---mixed with
a flexible background model---to fit a numerical model of a tidal
stream to observed stars in the vicinity of Palomar 5 Stream.  We find
that the Milky Way is XXX and the Stream is YYY.
\end{abstract}

\begin{eqnarray}
p(Y_n\given y_k,\theta,I) &=& N(Y_n\given y_k,\sigma^2_n+\Sigma^2)
\\
\theta &\equiv& \setof{\theta_\dyn, \theta_\ic, \Sigma^2}
\\
I &\equiv& \setof{K, \setof{\sigma^2_n}_{n=1}^N, \cdots}
\\
p(Y_n\given\theta,I) &=& \sum_{k=1}^K P_k\,p(Y_n\given y_k,\theta)
\\
1 &=& \sum_{k=1}^K P_k
\\
P_k &=& \frac{1}{K}
\\
P(\setof{Y_n}_{n=1}^N\given\theta,I) &=& \prod_{n=1}^N p(Y_n\given\theta,I)
\\
p(Y_n\given\theta,I) &=& P_\bg\,p(Y_n\given \bg,\theta) + \sum_{k=1}^K P_k\,p(Y_n\given y_k,\theta)
\\
\theta &\equiv& \setof{\theta_\dyn, \theta_\ic, \Sigma^2, P_\bg, \theta_\bg}
\\
1 &=& P_\bg + \sum_{k=1}^K P_k
\\
P_k &=& \frac{[1 - P_\bg]}{K}
\end{eqnarray}

\end{document}
